\chapter{Method}

To recap, the question behind this thesis is:
\begin{quote}
Is it possible to create an engaging virtual environment in which Swedish gymnasium students can learn effectively, according to constructionist learning theory?
\end{quote}

Now that we know a little background to microworlds and constructionist learning theory, we have to come up with a method for constructing and evaluating the virtual environment (or microworld) named in the question above.

\subsection{Constraints on my method}

My method should produce something that satisfies the following constraints:
\begin{itemize}
  \item It should allow students to come up with their own \textbf{ideas}.
  \item It should allow students to \textbf{implement} their ideas in their own way.
  \item It should allow students to \textbf{evaluate} the implementation of their ideas.
\end{itemize}

From our original aims, our produced microworld should also\ldots
\begin{itemize}
  \item \ldots be suitable for classroom-use by gymnasium students in Sweden.
  \item \ldots encourage students to create ``public entities''.
\end{itemize}

\subsection{Choice of implementation}

I choose to fulfil the constraints listed above by creating \textbf{web applications that serve as in-browser microworlds}. I will design these web applications specifically for the classroom. Their purpose is to enable students to create digital public entities. I do this because I have a lot of experience in building web applications.

\subsection{How to evaluate my implementation}

 I want to evaluate the \textit{process} by which students create their public entities, not those entities themselves. Exploring microworlds can be seen as a form of role-play, meaning that I can evaluate this process as a simulation exercise.

 According to Megarry (1978)\cite[p187-207]{megarry}, simulations can be evaluated in the following ways:

\begin{itemize} 
  \item ``using narrative reports''
  \item ``using checklists gathered from students' recollections of outstanding positive and negative learning experiences''
  \item ``encouraging players to relate ideas and concepts learned in games to other areas of their lives''
  \item ``using the instructional interview, a form of tutorial carried out earlier with an individual learner or small group in which materials and methods are tested by an instructor who is versed not only in the use of the materials, but also in the ways in which pupils learn.''
\end{itemize}

I believe checklists are too simplistic a form of evaluation method, and I consider instructional interviews too time-consuming. Again, I don't want to evaluate the public entities themselves - treating them as narrative reports - because a guideline for evaluating simulations is that you should be ``primarily concerned with the process rather than the product of simulation''\cite{lindabook}.

Consequently, I choose the third item above: ``encouraging players to relate ideas and concepts learned in games to other areas of their lives''. This will take the form of oral group discussions of texts, which will be informally evaluated by myself and the teacher present during the workshop.
