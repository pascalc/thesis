\section{Choosing a method}

To recap, the question behind this thesis is:
\begin{quote}
Is it possible to create an engaging virtual environment in which Swedish gymnasium students can learn effectively, according to constructionist learning theory?
\end{quote}

Now that we know a little background to microworlds and constructionist learning theory, we have to come up with a method for creating and evaluating the virtual environment (or microworld) named in the question above.

\subsection{Constraints on our method}

Our method should produce something that satisfies the following constraints:
\begin{itemize}
  \item It should allow students to come up with their own \textbf{ideas}.
  \item It should allow students to \textbf{implement} their ideas in their own way.
  \item It should allow students to \textbf{evaluate} the implementation of their ideas.
\end{itemize}

From our original aims, our produced microworld should also\ldots
\begin{itemize}
  \item \ldots be suitable for classroom-use by gymnasium students in Sweden.
  \item \ldots encourage students to create ``public entities''.
\end{itemize}

\subsection{Method outline}

The following is an outline of the methods used:
\begin{itemize}
  \item I will create microworlds in the browser, so that students can run them without having to install additional software.
  \item I will design my software for classroom use.
  \item The purpose of my software will be for students to create their own public entities.
\end{itemize}

\subsection{Evaluating my methods}

In the end, I created two microworlds: \textit{Talking to Machines} (TTM) and \textit{Narrative Roulette} (NR). They will be described in detail in the following two sections. 

For now, it is sufficient to say that TTM is a microworld in the domain of programming and geometry, and NR is a microworld in the domain of Natural Language and role-play.

\subsection{Evaluating \textit{Talking to Machines}}

How does one evaluate a programming microworld? The public entities produced in a programming microworld are computer programs. I planned to use peer-evaluation to evaluate the quality of these programs, by having a students' peers give their opinion on the results of a program.

Unfortunately, TTM was too complex a microworld, in terms of syntax and semantics, to be suitable for classroom-use by gymnasium students. I established this first-hand by introducing non-programmers to the microworld, and realising that I had to provide considerable help in order for them to produce even a working program.

This meant that I needed to create another microworld, which would be suitable for classroom-use by gymnasium students: NR.

\subsection{Evaluating \textit{Narrative Roulette}}

The public entities created in NR were fictional texts in which students' took on certain perspectives. This means that these texts are the products of a simulation exercise. According to Megarry (1978)\cite[p187-207]{megarry}, simulations can be evaluated in the following ways:

\begin{itemize} 
  \item ``using narrative reports''
  \item ``using checklists gathered from students' recollections of outstanding positive and negative learning experiences''
  \item ``encouraging players to relate ideas and concepts learned in games to other areas of their lives''
  \item ``using the instructional interview, a form of tutorial carried out earlier with an individual learner or small group in which materials and methods are tested by an instructor who is versed not only in the use of the materials, but also in the ways in which pupils learn.''
\end{itemize}

I considered checklists too simplistic and instructional interviews too time-consuming as evaluation methods. I also didn't want to evaluate the texts themselves - treating them as narrative reports - because a guideline for evaluating simulations is that you should be``primarily concerned with the process rather than the product of simulation''\cite{lindabook}.

Consequently, I chose the third item above: ``encouraging players to relate ideas and concepts learned in games to other areas of their lives''. This took the form of oral group discussions of texts, which were informally evaluated by myself and the teacher present during the workshop.
