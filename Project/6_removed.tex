\emph{Talking to Machines} offered learning about two topics: the syntax
of a programming language, and hands-on knowledge of computation. These
are distinct: a programmer can know that she wants to write a recursive
loop, but not know the syntax for this in JavaScript.

Hands-on knowledge about computation is hard to acquire without knowing
a little syntax in at least one programming language. This is because if
the \textbf{implementation} of your computational \textbf{idea} (such as
how to print only odd integers less than 100) can only be expressed in
pseudocode, you will only be able to \textbf{evaluate} your idea by
writing it out by hand or asking a teacher if you are correct.

Being able to write in the syntax of a programming language means being
able to \textbf{implement} your computational \textbf{ideas} in a form
that can be instantly \textbf{evaluated} by a machine. The advantage of this is that you can learn things in an effective, engaging way
without having to ask for the permission or judgement of a human teacher
or academic institution.

The downside is that you need to learn that syntax before you can reap
the rewards of machine evaluation. 
