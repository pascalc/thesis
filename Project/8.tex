\section{Narrative Roulette on the Web}

A problem with \emph{Narrative Roulette} was that it required a
classroom and lesson-time for students to take part.

I managed to obtain a classroom without taking time from students on one
occasion, by holding a voluntary \emph{Narrative Roulette} workshop on
21 January 2014 at Kungsholmens Gymnasium. Turnout was better than
expected (15 students and friends attended), and the session proved that
the workshop would work. But it wasn't a true proof of the concept until
it was tried on students that \emph{had} to be there as part of their
ordinary school day.

Finding lesson-time for \emph{Narrative Roulette} workshops was
extremely difficult. Teachers, understandably, seemed to prefer to spend
their lesson time achieving the goals of their curricula directly, with
their own lesson plans, rather than taking a risk on the indirect value
provided by \emph{Narrative Roulette}.

A potential solution to this problem was to remove the dependency on the
physical classroom by moving to a purely virtual environment, and trying to take a share of students' leisure time instead of their school-time.

\subsection{NarrativeRoulette.com}

I registered the domain \url{http://narrativeroulette.com} and modified
\emph{Narrative Roulette} so that rounds could take place online.

The modifications were the following: 

\begin{itemize}

\item Now, rounds would last for one
week instead of 15 minutes, with submissions accepted at any time that
week.

\item All submissions were anonymously posted to Facebook, by the
Narrative Roulette Facebook Page (\href{https://www.facebook.com/narrativeroulette}{https://www.facebook.com/narrativeroulette}).

\item Discussion took place in Facebook comments. 

\item Stories could be shared via Facebook. 

\item Communication between myself and Narrative Roulette participants took place via posts on the Narrative Roulette Facebook page.

\end{itemize}

\subsection{Is this still a microworld?}

An iteration of the microworld loop in NarrativeRoulette.com looks like
this: 

\begin{enumerate}

\item The user reads the perspective of the current round. For
example: ``You are female. You are more scared than you've ever been in
your life.'' 

\item The user writes a short narrative, from that
perspective, using the editor at NarrativeRoulette.com. This is the
\textbf{idea} stage of the microworld loop. 

\item The user submits their
narrative. This is the final \textbf{implementation} stage of the
microworld loop. 

\item People who have liked Narrative Roulette on Facebook
see the user's text published by Narrative Roulette. They
\textbf{evaluate} the text by liking, sharing and commenting on it, all
in Facebook.

\end{enumerate}

As we can see, in the NarrativeRoulette.com microworld, the
\textbf{idea} and \textbf{implementation} stages of the microworld loop
stayed the same as in the classroom version. The changes were to the
\textbf{evaluation} stage, where texts were read, discussed and
evaluated on Facebook - a virtual space instead of a physical one.

\subsection{How did this work in practice?}

The NarrativeRoulette.com microworld was far less engaging than the
classroom version. A round on NarrativeRoulette.com received an average
of 7 submissions, while a round in the classroom version received as
many submissions as students in the classroom: around 30. This was
despite the reach of Narrative Roulette's Facebook page being 38 people.

I believe that there was also an indirect change to the \textbf{idea}
stage, and this is what caused the difference between the engaging power
of the two versions.

When you're in a classroom, attending a \emph{Narrative Roulette}
workshop, you have a clear goal: to write a narrative from a certain
perspective within a short timespan (15 minutes). You are motivated to
do this, even if no teacher keeps an eye on your screen, because you
know your peers are all doing this at the same time, and you want to add
to the work being created.

Online however, the communal aspect of writing is lost, leading to a loss of motivation. Now you're not writing because you have
to, or for your classmates; you're writing for the internet, because you
\emph{want} to. And you don't have to submit in the next fifteen
minutes, you can submit any time until the end of the week (if at all). This results in a huge lack of \emph{urgency}. 

The fun of evaluating texts seemed to disappear too. In the classroom,
readers didn't know exactly who had written a text, but they knew it was
someone sitting in that room. I believe that led them to be more
forgiving, knowing that the writer was one of their peers.

When anonymous submissions appeared on Facebook, readers had no idea
who had written them. I believe this led people to be less forgiving (they were just another text on the internet, written by someone they didn't know), which
meant that they were reluctant to share, like or comment on the texts, even though they did read them.

\subsection{NarrativeRoulette.com takeaways}

Moving \emph{Narrative Roulette} online removed the need for acquiring
classroom space and lesson time from students. However, it failed to be
as engaging - there was less work produced and less discussion generated
than \emph{Narrative Roulette}'s physical version.

This means that I consider NarrativeRoulette.com not worth pursuing as a
project in its current form; the physical version of \emph{Narrative
Roulette} is the superior microworld.
