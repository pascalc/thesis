\section{Microworlds are engaging}

TODO: Talk about instant feedback too!

\subsection{What is the key to motivation?}

According to Dan Pink, it's autonomy\cite{drive}. Roman Krznaric agrees, and uses
the following statistic to prove his point: 47\% of self-employed people
say they are ``very satisfied'' with their jobs, compared to only 17\% of those in regular employment\cite{krznaric}.

Here is the table from which he got his data\cite{joywork}:

\begin{center}
  \begin{tabular}{ | l | r | r | r | }
    \hline
    \textbf{\% of respondents} & \textbf{Full-time} & \textbf{Part-time} & \textbf{Self-employed} \\ \hline
    Very dissatisfied & 5.8 & 3.2 & 1.6 \\
    Dissatisfied & 11.2 & 10.5 & 4.7 \\
    Neutral & 18.7 & 18.9 & 12.5 \\
    Satisfied & 46.9 & 48.4 & 34.4 \\
    Very satisfied & 17.3 & 18.9 & 46.9 \\
    \hline
  \end{tabular}
\end{center}

There are also more than twice as many dissatisfied people working full-time
compared to those who are self-employed.

Why would this be?

The report has this to say:

\begin{quote}
``This could be attributed to the control that the self-employed have over
their work: whilst many work very long hours, their ability to determine
when, where and how they work may contribute to their high levels of
satisfaction''\cite{joywork}.
\end{quote}

If you have the ``ability to determing when, where and how'' you work,
that means you have autonomy over your work.

\subsection{Do microworlds provide autonomy?}

The first stage of our microworld-defining loop is coming up with an
\textbf{idea} that obeys the constraints of the \emph{structure} of that
microworld.

Imagine you are a self-employed, freelance web designer. You have been
tasked with coming up with a new landing page for a coffee shop brand.

You have autonomy over when you work, where you work, and how you work.
You can work only between 22.00 and 04.00. You can work from home. You
can use or skeumorphic design, or flat design.

Let's say you try skeumorphic design, but after you \textbf{implement}
some skeumorphic elements, and \textbf{evaluate} them, something seems
off. You update your \textbf{ideas} and try flat design instead. Much
better.

You exercised your autonomy in \emph{how} you work.

But however much you iterate, you still need to deliver something that
looks like a web page. You can't deliver a design for a printed book. Or
a design for a 15-minute movie.

Those are the constraints of the \emph{structure} of your web design
microworld.

\subsection{What would a microworld look like without autonomy?}

Let's imagine our web design microworld without autonomy.

Your task is still the same: creating a landing page for a coffee shop
brand.

But this time, you must work between 09.00 and 17.00 on weekdays. You
must work from the office, from your cubicle. You're free to use
skeumorphic design or flat design\ldots{} but your boss \emph{really}
likes skeumorphic design.

After \textbf{implementing} your initial skeumorphic design
\textbf{idea}, you \textbf{evaluate} it, and as before, you find it
lacking.

Unfortunately, you can't change it. Because your boss likes it, and
their opinion overrules yours.

The structural constraints of the microworld still apply. But now some
arbitrary constraints \emph{also} apply: over where, when and how you
work.

These additional, arbitrary constraints suck the fun right out of the
microworld. They make the microworld less engaging. They make it feel
like work, not a game.

\subsection{What does engagement lead to?}

When a microworld allows for autonomy, it is a highly motivating place
to experiment in.

In Minecraft, this motivation leads to people building their first basic
block structure. For a band jamming together, or a novice writer, it
might lead to the first work they're comfortable sharing with friends.
For the budding programmer, it usually leads to their first non-trivial,
mostly-working program.

And that leads to a sense of accomplishment. But that's not where it
ends.

After an initial success, the temptation is there to tweak the original,
ever so slightly. The first block structure leads to a house; one song,
story or program leads to another, similar in structure but differing in
detail.

Eventually, we end up with a blockrealm the size of Los Angeles; a band
and author with a string of hits; a programmer in charge of a program
that serves billions around the world.

As we argued in the previous section, each iteration of our loop led to
learning for all involved.

But it took the engagement enabled by autonomy to start the loop
spinning in the first place.
