\section{What kinds of microworld exist today?}

By the age of 21, the average young person will have spent as much time
playing video games as they will have spent in a classroom \{ted\}.
Their presence in the classroom is required by law. Their presence in
front of the screen is entirely voluntary.

\subsection{What is a game?}

So what exactly is a game, and what lets them demand such attention from
children and young adults?

According to Jane McGonigal, a game designer and writer, the four
defining traits of a game are the following:

\begin{enumerate}[1.]
\item
  The \textbf{goal} is the specific outcome that players will work to
  achieve. It focuses their attention and continually orients their
  participation throughout the game. The goal provides players with a
  sense of purpose.
\item
  The \textbf{rules} place limitations on how players can achieve the
  goal. By removing or limiting the obvious ways of getting to the goal,
  the rules push players to explore previously uncharted possibility
  spaces. They unleash creativity and foster strategic thinking.
\item
  The \textbf{feedback system} tells players how close they are to
  achieving the goal. It can take the form of points, levels, a score,
  or a progress bar. Or, in its most basic form, the feedback system can
  be as simple as the players' knowledge of an objective outcome: ``The
  game is over when . . .'' Real-time feedback serves as a promise to
  the players that the goal is definitely achievable, and it provides
  motivation to keep playing.
\item
  Finally, \textbf{voluntary participation} requires that everyone who
  is playing the game knowingly and willingly accepts the goal, the
  rules, and the feedback. Knowingness establishes common ground for
  multiple people to play together. And the freedom to enter or leave a
  game at will ensures that intentionally stressful and challenging work
  is experienced as \emph{safe} and \emph{pleasurable} activity.
  \emph{Taken from ``Reality is Broken''\{rebroken\}, by Jane McGonigal,
  emphasis hers.}
\end{enumerate}

To summarise, a game comprises of the \textbf{voluntary} attempt to
achieve a certain \textbf{goal}, according to a set of \textbf{rules},
with \textbf{feedback} on how close you are to that goal.

\subsection{Are games microworlds? Or are microworlds games?}

Let's compare this with our definition of a microworld, from the
previous section. A microworld is an environment that supports a loop
comprising of:

\begin{enumerate}[1.]
\item
  An \textbf{idea} that a student comes up with \emph{themselves}.
\item
  An \textbf{implementation} of that idea, according to the \emph{rules}
  of the microworld.
\item
  An \textbf{evaluation} of the implementation, that gives the student
  \emph{feedback} about the quality of their implementation and idea.
\item
  The evaluation generates further ideas, and the loop continues.
\end{enumerate}

As we can see, there seems to be an isomorphism between microworlds and
games. An \textbf{idea} in a microworld corresponds to a \textbf{goal}
that a student chooses \textbf{voluntarily}. The \textbf{implementation}
of that idea must conform to certain \textbf{rules}, set by the
\emph{structure} of the microworld. And the purpose of
\textbf{evaluating} a student's implementation is to give them
\textbf{feedback} about it, so they can improve their future ideas.

But the two are not exactly equivalent, as a microworld's \textbf{idea}
conflates the \textbf{voluntary} and \textbf{goal} traits of a game.
Every microworld is a game (each microworld contains all four traits of
a game), but not all games are microworlds.

\begin{quote}
Only those games that allow the voluntary choice of goals can be
considered microworlds.
\end{quote}

\subsection{Flappy Bird vs Minecraft}

Wikipedia says this about the gaming phenomenon known as Flappy Bird
\{3\}:

\begin{quote}
Flappy Bird is a side-scrolling mobile game featuring 2D retro style
graphics. The objective is to direct a flying bird, which moves
continuously to the right, between each oncoming set of pipes without
colliding with them, which otherwise ends the game. The bird briefly
flaps upward each time the player taps the screen. If the screen is not
tapped, the bird falls due to gravity. The player is scored on the
number of pipe sets the bird successfully passes through, with medals
awarded for the score.
\end{quote}

Though players play Flappy Bird \textbf{voluntarily}, they have no say
in their \textbf{goal}. They just have to keep flapping, or they die and
the game is over. The simplicity of Flappy Bird's \textbf{rules} (just
flap) and the sophistication of its \textbf{feedback system} (if I'd
flapped \emph{slightly} earlier, I'd be alive!) are what make the game
addictive \{forbes\}.

This means that Flappy Bird qualifies as a game, but not as a
microworld.

Minecraft, on the other hand, is defined by Wikipedia as:

\begin{quote}
Minecraft allow{[}s{]} players to build constructions out of textured
cubes in a 3D procedurally generated world. Other activities in the game
include exploration, gathering resources, crafting, and combat. Gameplay
in its commercial release has two principal modes: survival, which
requires players to acquire resources and maintain their health and
hunger; and creative, where players have an unlimited supply of
resources, the ability to fly, and no health or hunger.
\end{quote}

The goal of Minecraft's \textbf{survival} mode is, unsurprisingly, to
survive. This goal is non-negotiable. However, Minecraft also has
another mode, in which players have neither health nor hunger. This
means that there is only one reason for playing this mode of the game:
the joy of building things from virtual, textured cubes.

In this particular case, of a specific mode of a specific game, we have
an environment in which players can \emph{choose their goals with total
freedom}. This makes Minecraft's \textbf{creative} mode a model example
of something that is both a game and a microworld.

\subsection{What other activities can be microworlds?}

We've established that microworlds are a subset of games - those
activities that contain all four of McGonigal's traits listed above. Of
course, the set of all games is wider than just video games. Let's look
at some other activities that could be described as microworlds.

\subsubsection{Jamming}

Consider a few people gathering in a room, each with their own musical
instrument, jamming together. There is a \textbf{goal}: to make music
that sounds good. There are \textbf{rules} (or \emph{constraints}): of
timing (4 beat bars), chord sequences that sound good together, the
traditions of the genre, the expectations of the potential audience,
etc. \textbf{Feedback} is instant: the musicians can tell if something
sounds good, \emph{while they're playing it}. And of course, jamming is
\textbf{voluntary} in the vast majority of cases.

This means that jamming is actually a game, even though we wouldn't
usually describe it that way. Is jamming also a microworld?

For a game to also be a microworld, it has to allow voluntary choice of
\textbf{goals}. Though the overarching goal of jamming is to make good
music, the group can work towards that with subgoals.

Perhaps the group decide the first step is to come up with a catchy
chorus. The guitarist might have some ideas for chords she wants to try;
the singer has some idea of what words he wants to sing; the drummer has
some ideas about a beat, and so on.

These \textbf{ideas} are \textbf{implemented} immediately, and
\textbf{evaluated} soon after, by the musicians who judge whether what
they're doing is working. If it isn't, they update their \textbf{ideas}:
the drummer tries a different beat, or the guitarist switches chords,
and they iterate. Otherwise, they move on to another part of the song.

In this light, jamming can be described as a microworld, as it shares
its basic structure with something like Minecraft, even though, on the
surface, the two activities look very different.

\subsubsection{Creative writing}

Writing fiction is a microworld too. A lone writer has an \textbf{idea}
in her head for a story. It probably isn't the whole story, word for
word, fully-formed in her head. It's probably a rough idea, like:
``dinosaurs on the loose in zoo, people try to escape'' or ``criminal
couple rob banks and are then shot by the police''.

The smaller ideas that make up this grand vision might be: ``the people
escaping the dinosaurs are archaeologists'', or ``wouldn't it be cool to
have a scene where a rampaging tyrannosaurus rex destroys a skeleton of
its own species?''.

The \textbf{implementation} of these ideas takes the macro form of
scenes, characters and events; and the micro form of the words used to
describe them. \textbf{Evaluation} occurs when the writer reads back
what she's written, a process that the author David Foster Wallace once
called ``feed{[}ing{]} the wastebasket''\{flesh\}.

Again, this makes writing a game, and the free choice of goals, or
ideas, makes it a microworld too.

\subsubsection{Programming}

The microworld of programming is the one I have most experience with. An
\textbf{idea} in this world could be something like: ``I want to build a
system where users can vote for things''. Smaller ideas could be: ``the
system should work on mobile phones'' and ``the page should update by
itself''.

An \textbf{implementation} could use a HTML5 front-end, a Python
back-end, and HTTP and WebSockets to communicate between them.

The \textbf{evaluation} would be testing the system, first in unit and
end-to-end automated tests, and finally by real users in a production
environment.

Iterating this loop at speed is known as \textbf{extreme
programming}\{wiki:extreme\} (although as with all programming jargon,
its exact meaning is hard to pin down).

\textless{}\textless{} THE REPL IS A MICROWORLD
\textgreater{}\textgreater{}

\subsection{Degrees of Microworld}

So we have a bunch of environments that we think are microworlds:
sandbox video games (like Minecraft), jamming, creative writing and
programming. They all seem to be games (or at least have the potential
to be), and as ideas can be chosen freely, we argue that they are
microworlds too.

But how ``free'' are they, and if we gradually restrict this freedom,
will we eventually end up with things that are no longer microworlds?

Let's take our programming example. Imagine the project is a commercial
one, paid for by a client with certain requirements. These requirements
include: * The front-end must look exactly like the given mockups. * The
front-end must be an iOS app. * The back-end must be written in Java,
and use servlets. * The entire project must be production ready within a
week.

Do we still have a microworld? The person or team working on the project
is severely restricted in what \textbf{ideas} they can have. They have
to conform to the mocked-up design, write it in the specified front and
back-end languages, and can only do things that will fit in the given
time-frame.

As a result, the solution space is extremely narrow; there's probably
only a few ways to do anything, and maybe only one that could work given
all the constraints. So the team only really \textbf{implement}, and the
client is the one who will \textbf{evaluate}.

Just implementing does not a microworld make. It doesn't even qualify as
a game, as participation is not \textbf{voluntary}, and the
\textbf{feedback system} is probably far from real-time.

\begin{quote}
The more freedom allowed in an activity, the more of a microworld it is.
\end{quote}

Ideally, the only constraints are those imposed by the structure of the
microworld: in programming we are limited by our technology; in writing
or music we are limited by our own abilities and the culture in which we
live.

But an activity can still be a microworld, even if we have a deadline
and have some constraints over those imposed by the environment. Writing
a story can still be a microworld even if it has to be done by the end
of the month, and must include dinosaurs and a happy ending.

Because the solution space is still wide-open; \textbf{ideas} can still
be had. But once scenes, characters and actions start being heavily
constrained, the potential for \textbf{ideas} will reduce, and the
microworld will become just another job.

NEXT: So we know what microworlds are, and which ones exist today. What
evidence do we have that they enable learning?

\begin{center}\rule{3in}{0.4pt}\end{center}

\{ted\} Jane McGonigal
\href{http://www.ted.com/conversations/44/we\_spend\_3\_billion\_hours\_a\_wee.html}{http://www.ted.com/conversations/44/we\_spend\_3\_billion\_hours\_a\_wee.html}
\{rebroken\} Jane McGonigal ``Reality is Broken'' \{wiki:flappy\}
Wikipedia Flappy Bird \{forbes\}
http://www.forbes.com/sites/ewanspence/2014/02/18/the-vital-and-depressing-lessons-flappy-bird-can-teach-indie-developers/
\{flesh\} Both Flesh And Not, David Foster Wallace \{wiki:extreme\}
Wikipedia Extreme Programming
