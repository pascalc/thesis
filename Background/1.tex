\section{What is a microworld?}

Imagine you're really hungry. It's late in the day and you need to
decide what you're doing for dinner. This is a problem faced by most of
us, repeated day after day.

What does the solution space look like? The usual way to figure this out
is by examining a single solution and then seeing how its properties
might vary.

A single solution to the whats-for-dinner problem is a single meal. So
the solution space encompasses everything that can be considered a
single meal: everything from a steak to a snack bar. And of course
there's the null solution: not to eat anything at all.

\subsection{Open-ended goals: Creativity within constraints. Coming up
with a \textbf{creative idea}.}

We explore this solution space every time we plan a meal. We work under
certain constraints, such as availability, our personal preferences,
diet plans and even the cultural acceptance of the meal in question. The
set of things we are likely to eat is far smaller than the set of all
things that we \emph{could} eat.

That said, we can also be creative with our solutions. Though our goals
must conform to a certain shape (say roughly 1000 calories of
nutrition), they are also \textbf{open-ended}. There are an infinite
number of ways to get 1000 calories of nutrition, even taking into
account all of the constraints we just mentioned.

Of course, we rarely think about all of these possibilites. We tend to
stick to what we know, reducing an infinite vastness to the comfortably
familiar.

I cook pasta a \emph{lot}, and I doubt I'm the only one.

If I decide to combine pasta with bacon and pesto, I have come up with a
potential solution to the whats-for-dinner problem. I call this a
creative idea in the solution space.

To realise my creative idea, I have to cook.

\subsection{Implementation}

How will I fry the bacon? Shall I make pesto myself or buy it in a jar?
What kind of pasta shall I use, and how should I prepare it?

When I make these decisions, I am trying to realise my abstract idea
with a concrete implementation. I am turning a creative idea in my head
into something I can put on my plate.

\subsection{Evaluation}

Once my reified idea is on my plate, the evaluation phase begins. If my
solution contains roughly 1000 calories, we can call it a meal. But is
it a satisfactory one?

Like the solution space, the evaluation space is infinite. I may have
achieved my nutritional goals, but failed my goals of taste. Or maybe my
meal was both nutritious and tasty, but it just wasn't \emph{novel}
enough.

We might call judging a meal hopelessly subjective. One person's
delicacy is another person's travesty. Yet certain dishes achieve global
popularity. The owners of the most famous restaurants probably believe
that the popularity of a certain dish can be reliably predicted.

\subsection{Feedback and iteration}

An evaluation of the implementation of a creative idea is only useful if
it guides future actions.

Imagine I feel that my home-made pesto failed utterly. This is only a
productive thought if the next time I plan a meal, I change the way I
acquire pesto. Either I change how I make it, or I buy it ready-made
from the supermarket. The success of my updated method will only be seen
when I taste my new, modified meal.

In this way I can learn about what succeeds and what fails.

\subsection{Learning as a process}

\textbf{Learning is what happens when someone executes an
idea-implementation-evaluation loop, and uses outcomes to guide future
iterations.}

\textbf{A microworld is an environment that enables a person to iterate
an idea-implementation-evaluation loop within a certain domain.}

A supermarket, kitchen and dinner table is a microworld for meals.
