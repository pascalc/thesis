\section{What is a microworld?}

Imagine this. You're hungry. It's late in the day and you need to
decide what you're doing for dinner. It's a familiar problem.

What does the solution space for this problem look like? The usual way to figure this out
is by examining a single solution and then seeing how its properties vary.

A single solution to the whats-for-dinner problem is a single meal. So
the solution space encompasses everything that can be considered a
single meal: everything from a steak to a snack bar. And of course
there's the null solution: not to eat anything at all.

\subsection{Ideas: open-ended goals}

We explore this solution space every time we plan a meal. We work under
certain constraints, such as the availability of ingredients, our personal preferences,
diet plans and even the cultural acceptability of certain meals. The
set of things we are likely to eat is far smaller than the set of all
things that we \emph{could} eat.

That said, we can also be creative with our solutions. Though our goals
must conform to a certain shape (say roughly 1000 calories of
nutrition), they are also \textbf{open-ended}. There are an infinite
number of ways to get 1000 calories of nutrition, even taking into
account all of the constraints we just mentioned.

Of course, we rarely think about all of these possibilites. We tend to
stick to what we know, reducing an infinite vastness to the comfortably
familiar.

I cook pasta dishes a \emph{lot}, and I doubt I'm the only one.

If I decide to combine pasta with meat and pesto, I have come up with a
potential solution to the whats-for-dinner problem. This is a
creative idea in the domain.

To implement my creative idea, I have to cook.

\subsection{Implementation}

What kind of meat shall I use? Shall I make pesto myself or buy it in a jar?
What kind of pasta shall I use, and how should I prepare it?

When I make these decisions, I realise my abstract idea
with a concrete implementation. I turn the creative idea in my head
into something I can put on my plate. 

In this case, let's imagine that the idea of pasta is reified into boiled, fusili pasta; meat is reified as fried bacon; and pesto is reified as the home-made kind. 

\subsection{Evaluation}

Once my reified idea is on my plate, the evaluation phase begins. If my
solution contains roughly 1000 calories, we can call it a meal. But is
it a satisfactory one?

Like the solution space, the evaluation space is infinite. I may have
achieved my nutritional goals, but failed my goals of taste. Or maybe my
meal was both nutritious and tasty, but it just wasn't \emph{novel}
enough.

In the end, my evaluation of my meal is evident in what I do the next time I try to solve the whats-for-dinner problem. If I prepare the pasta and the bacon in the same way, but change how I make the pesto, then evidently my pasta and bacon implementations were good enough, but I felt my pesto implementation could be improved. 

\subsection{Learning is a process}

By \textbf{implementing} my \textbf{creative idea} as a consumable ``product'', and then \textbf{evaluating} it by consuming it, I have \textit{learnt} about what works and what doesn't in the whats-for-dinner domain. I have learnt that I can make tasty pasta and bacon. I have learnt that my pesto needs improvement, and my first attempt should have given me some hints about what I should try next time. 

\begin{quote}
  Learning is what happens when someone executes an
  \textbf{idea-implementation-evaluation loop}, and uses feedback to guide future
  iterations.
\end{quote}

Seymour Papert coined the term ``microworld'' in 1980, describing them as ``incubators for knowledge''\cite[p120]{mindstorms}.

I propose the following, more concrete, definition:

\begin{quote}
  \textbf{A microworld is an environment that enables a person to iterate
  an idea-implementation-evaluation loop within a certain domain.}
\end{quote}

A supermarket, kitchen and dinner table is a microworld for learning how to cook meals - for learning to solve the whats-for-dinner problem, day after day.
