\section{Microworlds enable learning}

\subsection{John Dewey}

99 years ago, John Dewey, a philosopher of education, wrote:

\begin{quote}
``{[}\ldots{}{]} the school in turn will be a laboratory in which the
student of education sees theories and ideas demonstrated, tested,
criticized, enforced, and the evolution of new truths.''\cite[p55]{dewey}
\end{quote}

Dewey conceived of a school consisting of laboratories and studios,
where students would be able to construct things and experiment with
their own hands, instead of constantly being told what to do
and how to do it.

He wanted to see \textbf{ideas} demonstrated (or \textbf{implemented});
tested and criticized (or \textbf{evaluated}); for the evolution of new
truths (or the updating of ideas through iteration).

\subsection{Jean Piaget}

According to Edith Ackermann's analysis of the work of Jean Piaget:

\begin{quote}
``To Piaget, knowledge is not information to be delivered at one end, and
encoded, memorized, retrieved, and applied at the other end. Instead,
knowledge is experience that is acquired through interaction with the
world, people and things.''\cite{ackermann}
\end{quote}

This is because, to Piaget (according to Ackermann):

\begin{quote}
``Kids don't just take in what's being said. Instead, they interpret what
they hear in the light of their own knowledge and experience.''\cite{ackermann}
\end{quote}

Imagine an adult and child standing near a fire. According to
Ackermann's analysis of Piaget, Piaget would say that the child will not
refrain from touching the flame just because the adult tells the child
it will burn their hand (they ``don't just take in what's being said'').

Once the adult has gone, the child will reach for the flame regardless.
They will only abort their attempt to touch the fire once its heat on
their hand becomes uncomfortable (adding ``their own knowledge and experience'' to what they've been told).

Here is our first hint at \emph{why} microworlds could be useful for
learning. It's because microworlds are environments in which rich
experiences can be had (experiences such as those described by Dewey).
Microworlds are also safe places to have those experiences - hurting
yourself in a game is far less painful than doing so in real life.

\subsection{Seymour Papert}

\begin{quote}
``Constructionism means "Giving children good things to do so that they
can learn by doing much better than they could before." {[}\ldots{}{]}
Instructionism is the theory that says, "To get better education, we
must improve instruction."''\cite{convsinst}
\end{quote}

\begin{quote}
``Constructionism {[}\ldots{}{]} shares constructivism's connotation of
learning as "building knowledge structures" irrespective of the
circumstances of the learning. It then adds the idea that this happens
especially felicitously in a context where the learner is consciously
engaged in constructing a public entity, whether it's a sand castle on
the beach or a theory of the universe.''\cite{sitconst}
\end{quote}

Piaget thought that knowledge is \textbf{constructed} by the learner: we
build \textbf{knowledge structures} in our minds, from concrete
experiences - interactions with the world, people and things. This is called ``constructivism''.

Papert added to this theory, to come up with what he calls
``constructionism'' - the idea that knowledge structures are best
constructed when the learner is building a \emph{public} entity.

I believe the inclusion of the word ``public'' is important. I think it
has to do with the \textbf{evaluation} stage of our microworld loop.

If an entity is public, it will be evaluated in public, by more people
than its creator. This allows for richer feedback than if the creator
was the sole person in charge of evaluating their work.

In practice, this means that products of microworlds (whether they are
songs, stories or programs) should be exhibited to maximise the value of the feedback received by the creator of the work.

\begin{quote}
``Now one can make two kinds of scientific claim for constructionism. The
weak claim is that it suits some people better than other modes of
learning currently being used. The strong claim is that it is better for
everyone than the prevalent ``instructionist'' modes practiced in
schools. A variant of the strong claim is that this is the only
framework that has been proposed that allows the full range of
intellectual styles and preferences to each find a point of equilibrium.''\cite{sitconst}
\end{quote}

Microworlds allow students to construct knowledge \emph{in their own
way}. They offer building blocks that each student can use to construct knowledge in their own
personal style. 

The ``instructionist'' framework of traditional educational does not allow this. When you are being told how to do things, you need to follow the rules set by the instructor, not those that you come up with yourself.

\subsection{What do microworlds reward?}

Let's look at what microworlds reward in their users. To understand
this, we should look at the Papertian public entities that would be
constructed within these microworlds.

Consider Minecraft. Fans of the TV and book series \emph{Game of
Thrones} have built a version of Westeros, the series' fantasy realm,
from 1.2 billion bricks. Compared to the size of the characters, this
Minecraft construction is the size of Los Angeles\cite{westeroscraft}.

Creating something the size of Los Angeles, even if it only exists in
cyberspace, requires considerable skill. Builders need to be able to
place single blocks to form superstructures, and compose those
superstructures to form hyperstructures, and so on\cite{reddit:minecraft}.

When shared on servers, Minecraft artifacts are public entities. How are
these evaluated by other members of the public? Like other works of art.
It appears that complexity for its own sake is not admired; non-trivial
complexity must be twinned with aesthetic beauty for an artifact to be
admired\cite{mash}.

The same can be said of the creative writing microworld. Ralph Ellison, the American novelist, has said: ``Good
fiction is made of what is real, and reality is difficult to come by,''
which can be interpreted as saying that writing fiction rewards non-trivial ``truth'' value, instead of mere complexity.

Playing Flappy Bird, on the other hand, rewards (non-trivial) hand-eye coordination. A public entity in the
Flappy Bird world is a player's high score. Using this as a
feedback mechanism does increase your hand-eye coordination, in a very narrow domain,
but does not improve much else. 

\subsection{How do these rewards lead to learning?}

During iterations of the microworld loop in the microworlds mentioned, you do the following:

\begin{itemize}

\item You build complex superstructures in Minecraft\cite{local:minecraft}.
\item You think creatively and express yourself in natural language when writing fiction\cite{neural}.
\item You time your flaps in Flappy Bird.

\end{itemize}

There is a learning algorithm in the field of Artificial Intelligence called \textbf{Reinforcement Learning}. It can be described as follows:

\begin{quote}
``[An agent] must discover which actions yield the most reward by trying them. In the most interesting and challenging cases, actions may affect not only the immediate reward but also the next situation and, through that, all subsequent rewards.''\cite{reinforcement}
\end{quote}

By carrying out actions in its environment, evaluating rewards and iterating, an agent \emph{learns}. If we accept that this applies to human agents as well as artificial ones, we must conclude that iterating the microworld loop within microworlds enables learning too.

Therefore, the rewards in the environments mentioned lead to learning in the following ways:

\begin{itemize}

\item Building in Minecraft makes you better at creating complex superstructures.
\item Writing fiction makes you better at thinking creatively and expressing yourself.
\item You get better at timing your flaps in Flappy Bird.

\end{itemize}

The first two environments, Minecraft and creative writing, seem to lead to \emph{deeper} learning than Flappy Bird. This is because they are microworlds as well as being games. That said, the mechanism by which learning takes place is the same - reinforcement learning.
