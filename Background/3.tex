\section{Microworlds enable learning}

\subsection{John Dewey}

112 years ago, John Dewey, a philosopher of education, wrote:

\begin{quote}
{[}\ldots{}{]} the school in turn will be a laboratory in which the
student of education sees theories and ideas demonstrated, tested,
criticized, enforced, and the evolution of new truths. \{dewey:p55\}
\end{quote}

Dewey conceived of a school consisting of laboratories and studios,
where students would be able to construct things and experiment with
their own hands, instead of constantly being told to memorise what to do
and how to do it.

He wanted to see \textbf{ideas} demonstrated (or \textbf{implemented});
tested and criticized (or \textbf{evaluated}); for the evolution of new
truths (or the updating of ideas through iteration).

\subsection{Jean Piaget}

According to Edith Ackermann's analysis of the work of Jean Piaget:

\begin{quote}
To Piaget, knowledge is not information to be delivered at one end, and
encoded, memorized, retrieved, and applied at the other end. Instead,
knowledge is experience that is acquired through interaction with the
world, people and things. \{ackermann\}
\end{quote}

This is because, to Piaget (according to Ackermann):

\begin{quote}
Kids don't just take in what's being said. Instead, they interpret what
they hear in the light of their own knowledge and experience.
\{ackermann\}
\end{quote}

Imagine an adult and child standing near a fire. According to
Ackermann's analysis of Piaget, Piaget would say that the child will not
refrain from touching the flame just because the adult tells the child
it will burn their hand.

Once the adult has gone, the child will reach for the flame regardless.
They will only abort their attempt to touch the fire once its heat on
their hand becomes uncomfortable.

Here is our first hint at \emph{why} microworlds could be useful for
learning. It's because microworlds are environments in which rich
experiences can be had (experiences such as those described by Dewey).
Microworlds are also safe places to have those experiences - hurting
yourself in a game is far less painful than doing so in real life.

\subsection{Seymour Papert}

\begin{quote}
Constructionism means ``Giving children good things to do so that they
can learn by doing much better than they could before.'' {[}\ldots{}{]}
Instructionism is the theory that says, ``To get better education, we
must improve instruction.''\{convsinst\}
\end{quote}

\begin{quote}
Constructionism {[}\ldots{}{]} shares constructivism's connotation of
learning as ``building knowledge structures'' irrespective of the
circumstances of the learning. It then adds the idea that this happens
especially felicitously in a context where the learner is consciously
engaged in constructing a public entity, whether it's a sand castle on
the beach or a theory of the universe. \{sitconst\}
\end{quote}

Piaget thought that knowledge is \textbf{constructed} by the learner. We
build \textbf{knowledge structures} in our minds, from concrete
experiences - interactions with the world, people and things.

Papert added to this theory, to come up with what he calls
``constructivism'' - the idea that knowledge structures are best
constructed when the learner is building a \emph{public} entity.

I believe the inclusion of the word ``public'' is important. I think it
has to do with the \textbf{evaluation} stage of our microworld loop.

If an entity is public, it will be evaluated in public, by more people
than its creator. This allows for richer feedback than if the creator
was the sole person in charge of evaluating their work.

In practice, this means that products of microworlds (whether they are
songs, stories or programs) should be shared with at least one other
person than the creator, to be considered valuable.

\begin{quote}
Now one can make two kinds of scientific claim for constructionism. The
weak claim is that it suits some people better than other modes of
learning currently being used. The strong claim is that it is better for
everyone than the prevalent ``instructionist'' modes practiced in
schools. A variant of the strong claim is that this is the only
framework that has been proposed that allows the full range of
intellectual styles and preferences to each find a point of equilibrium.
\end{quote}

Microworlds allow students to construct knowledge \emph{in their own
way}. They offer building blocks that each student can use in their own
personal style. And the best way to showcase that style is in public.

According to Piaget, children learn by interpreting their experience in
their own way, by updating their own internal knowledge about a subject
in a \emph{constructive} way.

According to Papert, the best way to be construct knowledge is in the
process of constructing a public entity. This means that, learning to
draw is best achieved by actually drawing, according to constructivism,
but to satisfy constructionism, those drawings should be shown to
others.

When the drawings are shown, the creator gets feedback. They construct
the knowledge of how their ideas, mediated through their implementation,
is evaluated by other minds. They use this constructed knowledge to
explore new ideas.

\subsection{What do microworlds reward?}

Let's look at what microworlds reward in their users. To understand
this, we should look at the Papertian public entities that would be
constructed within these microworlds.

Consider Minecraft. Fans of the TV and book series \emph{Game of
Thrones} have built a version of Westeros, the series' fantasy realm,
from 1.2 billion bricks. Compared to the size of the characters, this
Minecraft construction is the size of Los Angeles\{westeroscraft\}.

Creating something the size of Los Angeles, even if it only exists in
cyberspace, requires considerable skill. Builders need to be able to
place single blocks to form superstructures, and compose those
superstructures to form hyperstructures, and so on\{reddit\}.

When shared on servers, Minecraft artifacts are public entities. How are
these evaluated by other members of the public? Like other works of art.
It appears that complexity for its own sake is not admired; non-trivial
complexity must be twinned with aesthetic beauty for an artifact to be
admired\{mash\}.

The same can be said of the creative writing and musical jamming
microworlds. Ralph Ellison, the American novelist, has said: ``Good
fiction is made of what is real, and reality is difficult to come by,''
which can be interpreted as saying that fiction is judged on non-trivial
complexity (``difficult to come by'') and truth value instead of
aesthetic beauty.

\textless{}\textless{} MUSIC - better when blends well
\textgreater{}\textgreater{}

Flappy Bird, on the other hand, rewards reflexes. A public entity in the
Flappy Bird framework is a player's point score. Using this is a
feedback mechanism does increase your reflexes, in a very narrow way,
but does not improve much else.

\subsection{Why do these rewards lead to learning?}

The feedback loops of the microworlds mentioned should do the following:
* Making things in Minecraft makes you better at building complex
structures. * Creative writing makes you better at expressing yourself
in writing, and writing the ``truth''. * Jamming makes you better at
blending musical styles.

We see this happen when AIs learn using Machine Learning
techniques\ldots{}

\begin{center}\rule{3in}{0.4pt}\end{center}

\{dewey\} The Child \& The Curriculum, John Dewey \{ackermann\}
http://learning.media.mit.edu/content/publications/EA.Piaget\%20\_\%20Papert.pdf
\{convsinst\} http://papert.org/articles/const\_inst/const\_inst1.html
\{sitconst\}
\href{}{http://www.papert.org/articles/SituatingConstructionism.html}
\{westeroscraft\}
\href{http://www.wired.com/2013/03/westeroscraft-game-thrones-minecraft/all/}{http://www.wired.com/2013/03/westeroscraft-game-thrones-minecraft/all/}
\{mash\}
\href{http://mashable.com/2013/02/13/amazing-minecraft-creations/\#gallery/25-minecraft-creations-that-will-blow-your-mind/53111bee12d2cd0acb004f80}{http://mashable.com/2013/02/13/amazing-minecraft-creations/\#gallery/25-minecraft-creations-that-will-blow-your-mind/53111bee12d2cd0acb004f80}
\{reddit\} http://www.reddit.com/r/Minecraft/comments/dibqq
