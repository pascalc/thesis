\section{Introduction}

In 2014, it was reported that Swedish 15-year-olds performed under the OECD-average in an international exam which tests problem-solving abilities\cite{skolverket:pisa}.

I believe that the low performance of Swedish students was not due to their lack of ability, but due to problems with their learning environment. In this thesis project I am going to try and create my own (virtual) learning environment, in which students can learn according to constructionist learning theory.

This can be formulated as the following question:
\begin{quote}
Is it possible to create an engaging virtual environment in which Swedish gymnasium students can learn effectively, according to constructionist learning theory?
\end{quote}

I will attempt to answer the question by constructing a virtual environment that I will let Swedish gymnasium students use. By observing how the system is used, and informally evaluating students' discussion of any work produced, together with an expert, I hope to find out if my virtual environment actually is engaging and effective.

My hope is that such an environment, if I manage to make one, can provide inspiration for educators that want to motivate students and enable their learning. 

\subsection{The scope of this project}

I choose to place the following constraints on my project:
\begin{itemize}
  \item The virtual environment should be suitable for classroom-use by gymnasium students in Sweden.
  \item The virtual environment should enable students to learn by personally constructing knowledge, according to constructionism.
  \item The virtual environment should encourage students to create ``public entities'' - products of their experimentation that can be shown to and evaluated by their peers.
\end{itemize}
