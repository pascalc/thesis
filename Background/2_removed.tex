\subsection{Degrees of Microworld}

So we have a bunch of environments that we think are microworlds:
sandbox video games (like Minecraft), jamming, creative writing and
programming. They all seem to be games (or at least have the potential
to be), and as ideas can be chosen freely, we argue that they are
microworlds too.

But how ``free'' are they, and if we gradually restrict this freedom,
will we eventually end up with things that are no longer microworlds?

Let's take our programming example. Imagine the project is a commercial
one, paid for by a client with certain requirements. These requirements
include: 

\begin{itemize}

\item The front-end must look exactly like the given mockups. 
\item The front-end must be an iOS app. 
\item The back-end must be written in Java and use servlets. 
\item The entire project must be production ready within a week.

\end{itemize}

Do we still have a microworld? The person or team working on the project
is severely restricted in what \textbf{ideas} they can have. They have
to conform to the mocked-up design, write it in the specified front and
back-end languages, and can only do things that will fit in the given
time-frame.

As a result, the solution space is extremely narrow: there's probably
only a few ways to do anything, and maybe only one that could work given
all the constraints. So the team only really \textbf{implement}, and the
client is the one who will \textbf{evaluate}.

Implementation in isolation does not make a microworld. It doesn't even qualify as
a game, as participation is not \textbf{voluntary}, and the
\textbf{feedback system} is probably far from real-time.

\begin{quote}
The more freedom allowed in an activity, the more of a microworld it is.
\end{quote}

Ideally, the only constraints are those imposed by the structure of the
microworld: in programming we are limited by our technology; in writing
or music we are limited by our own abilities and the culture in which we
live.

But an activity can still be a microworld, even if we have a deadline
and have some constraints beyond those imposed by the environment. Writing
a story can still be a microworld even if it has to be done by the end
of the month, and must include dinosaurs and a happy ending. Because the solution space is still wide-open, \textbf{ideas} can still
be had. But once scenes, characters and actions start being heavily
constrained, the potential for \textbf{ideas} will reduce, and the
microworld will disappear, leaving just another job.
