\subsection{Are there schools teaching microworlds right now?}

\subsubsection{Berghs School of Communication}

Berghs School of Communication, an educational institution in Stockholm, uses Action-Based Learning, a philosophy that is similar to microworlds as defined in this paper.

Read about their pedagogy \href{http://www.berghs.se/content/berghs-pedagogik}{here}\footnote{http://www.berghs.se/content/berghs-pedagogik} (Swedish). 

\subsubsection{Hyper Island}

Hyper Island, a global educational institution, uses ``constructivist methodology''.

Read about their philosophy \href{http://www.hyperisland.com/singapore/sgma/philosophy}{here}\footnote{http://www.hyperisland.com/singapore/sgma/philosophy}.

\subsection{How can the peer-evaluation of Narrative Roulette be
expanded to other microworlds?}

Works were evaluated in \emph{Narrative Roulette} by a student's peers,
based on number of readers, and those readers' sharing and discussion
behaviour.

Can this be extended to other microworlds?

The only domain-specific detail in the above evaluation method is
``reading''. This can be generalised to ``consumed'', or ``used'' for
other microworlds. For example, in a music-jamming microworld, we can
measure how many of a student's peers listen to produced songs; in a
programming microworld we can count how many other students interact at
length with a program. The sharing and discussing behaviour remain the
same.

A constraint is that only ``project work'' can be evaluated in this way.
This method is better for work in the form of a consumable product than in the form of solutions to
a problem set. This is entirely intentional: it forces us to
use microworlds to enable students to create ``public entities'', which
is a defining feature of constructivism.

\subsection{Should we drop lectures, exercise sessions and exams?}

This doesn't necessarily mean we should drop traditional learning
methods. Lectures and textbooks are still a very valuable source of
knowledge, and they are not at odds with the microworld paradigm.

Lectures and textbooks can be combined with microworlds when students
take part in \textbf{self-directed research}. This is what happens when
a student discovers a gap in their knowledge when exploring a
microworld. For example, in \emph{Talking to Machines}, a student might
want to learn about recursive loops, in order to draw concentric
circles. They could then read a chapter about recursive loops in a
textbook, or watch a lecture about them.

The difference from traditional teaching methods is \emph{why} a student
reads a book or watches a lecture. Now, the student is \emph{actively}
seeking information, instead of having it pushed at them and being told
to remember it. They can apply what they learn out of their own choice,
in their own way. Until a student does this, they have not
\emph{internalised} knowledge - made it their own.

Exercise sessions are subsumed by microworld exploration: a student
practices \textbf{implementation} every time they iterate the microworld
loop, so they don't need to work on practicing implementation in
isolation from having \textbf{ideas} and \textbf{evaluating} their work.

Traditional examinations could be entirely superseded by the peer-based
evaluation discussed above, but it's possible such a dramatic change in
the structure of education would be met with resistance. This is the
focus of the next section.

