\section{What do microworlds mean for education?}

Let's compare the structure of traditional teaching methods with
microworlds. Traditional teaching is often split into three activities:
lectures, exercise sessions, and exams. Lectures provide students with
theory, exercise sessions (or seminars) allow students to put that
theory into practice by solving problems, and exams allow students to
demonstrate both their theoretical and practical skills, in isolation -
test conditions, generally with no help from textbooks, the internet or
their peers.

There are rough parallels between these methods and microworlds.
Lectures correspond to \textbf{ideas} in a certain domain. A major
difference between ideas in lectures and ideas in microworlds is that
ideas in lectures are \emph{external}. They are other people's ideas,
such as ``Recursive loops never terminate if you leave out a
base-case''. In a microworld, the ideas you have are \emph{internal},
your own: ``If I write this base-case, my loop will terminate.''.

Exercises correspond to \textbf{implementation}. However, unlike
microworlds, traditional exercise sessions have students implementing
external ideas. For example: ``This is Pythagoras' theorem, now apply it
to this scenario!''. In a microworld, it is a student's internal ideas
that are implemented: ``I wonder if I can use Pythagoras' theorem to
help me draw a right-angled triangle.''

Exams are the \textbf{evaluation} stage. Unlike microworlds, exams do
not give instant feedback. Failing at problem-solving in an exam is not
a learning experience, it results in a period of uncertainty, which ends
when you receive a grade lower than you were expecting. If having your exam marked by a teacher
taught you something, perhaps how to solve a problem correctly, after
the course is over, you have no way to apply that knowledge, or to prove
that you now have it.

When you fail at problem solving in a microworld, you know immediately,
enabling you to update your understanding immediately. Evaluation is
\emph{productive} in a microworld, in a way that exams generally aren't.

In short, traditional learning methods do provide \textbf{ideas},
\textbf{implementation} and \textbf{evaluation}, but in components that
are spread out in time. A microworld places these components in a tight
loop that is iterated many times at speed. It is for this reason that
microworlds lead to more learning than traditional teaching methods.

\subsection{Agile methods vs The Waterfall Method}

There are echoes of this comparison in Software Engineering ideology.
Traditional teaching methods arguably correspond with The Waterfall
Method. The Waterfall Method consists of \cite{wiki:waterfall}: 

\begin{enumerate}
\item Requirements and Design (Microworld \textbf{idea}s). 
\item Implementation (Microworld \textbf{implementation}, obviously). 
\item Verification (Microworld \textbf{evaluation}).
\end{enumerate}

As in traditional teaching, these steps are separated in time. The
requirements and software design are done first, before any
implementation. Then the implementation is carried out. Finally, the
implementation is verified by the customer.

According to Dean Leffingwell and many others, ``the Waterfall Method
doesn't work''\cite{leffingwell}. Instead, most modern software engineers
prefer agile methods.

These agile methods move the requirements, design, implementation and
verification steps closer in time, and prefer fast iterations. Agile
methods make software engineering into a microworld.

\subsection{Are there schools teaching microworlds right now?}

\subsubsection{Berghs School of Communication}

Berghs School of Communication, an educational institution in Stockholm, uses Action-Based Learning, a philosophy that is similar to microworlds as defined in this paper.

Read about their pedagogy \href{http://www.berghs.se/content/berghs-pedagogik}{here} (Swedish). 

\subsubsection{Hyper Island}

Hyper Island, a global educational institution, uses ``constructivist methodology''.

Read about their philosophy \href{http://www.hyperisland.com/singapore/sgma/philosophy}{here}.

\subsection{How can the peer-evaluation of Narrative Roulette be
expanded to other microworlds?}

Works were evaluated in \emph{Narrative Roulette} by a student's peers,
based on number of readers, and those readers' sharing and discussion
behaviour.

Can this be extended to other microworlds?

The only domain-specific detail in the above evaluation method is
``reading''. This can be generalised to ``consumed'', or ``used'' for
other microworlds. For example, in a music-jamming microworld, we can
measure how many of a student's peers listen to produced songs; in a
programming microworld we can count how many other students interact at
length with a program. The sharing and discussing behaviour remain the
same.

A constraint is that only ``project work'' can be evaluated in this way.
This method is better for work in the form of a consumable product than in the form of solutions to
a problem set. This is entirely intentional: it forces us to
use microworlds to enable students to create ``public entities'', which
is a defining feature of constructivism.

\subsection{Should we drop lectures, exercise sessions and exams?}

This doesn't necessarily mean we should drop traditional learning
methods. Lectures and textbooks are still a very valuable source of
knowledge, and they are not at odds with the microworld paradigm.

Lectures and textbooks can be combined with microworlds when students
take part in \textbf{self-directed research}. This is what happens when
a student discovers a gap in their knowledge when exploring a
microworld. For example, in \emph{Talking to Machines}, a student might
want to learn about recursive loops, in order to draw concentric
circles. They could then read a chapter about recursive loops in a
textbook, or watch a lecture about them.

The difference from traditional teaching methods is \emph{why} a student
reads a book or watches a lecture. Now, the student is \emph{actively}
seeking information, instead of having it pushed at them and being told
to remember it. They can apply what they learn out of their own choice,
in their own way. Until a student does this, they have not
\emph{internalised} knowledge - made it their own.

Exercise sessions are subsumed by microworld exploration: a student
practices \textbf{implementation} every time they iterate the microworld
loop, so they don't need to work on practicing implementation in
isolation from having \textbf{ideas} and \textbf{evaluating} their work.

Traditional examinations could be entirely superseded by the peer-based
evaluation discussed above, but it's possible such a dramatic change in
the structure of education would be met with resistance. This is the
focus of the next section.

