\section{Conclusions}

Microworlds seem to enable learning in an engaging and effective way, by
allowing students to construct knowledge by exploring rich environments.
Unfortunately, there does not seem to be the infrastructure in place in
educational institutions today to fully embrace microworlds. Teachers do
not have the technical skills required to create or maintain virtual
microworlds, and politicians arguably have no incentive to allow them to
do so.

Teachers at Kungsholmens Gymnasium thought \emph{Narrative Roulette} was
a great idea, but could not find lesson-time for me to hold workshops
(due to the demands of the politican-set high-school curriculum). I only
managed to hold my three workshops when I worked together with a teacher
to have them during substitute lessons. And only then when those
substitute lessons came up too fast for the substitute teacher to be
able to produce a lesson plan for them. It was only in this very
particular scenario, when lessons appeared in which the actual
curriculum could not be followed, that \emph{Narrative Roulette} was
given a chance.

The students loved it\{ing:email\}. As I argued in Section 10, they also
learned a lot.

But only I could maintain the software during the workshops, and even if
I improved the interface so that others could run it, the workshops did
not provide clear, direct value for the current curriculum. This means
that \emph{Narrative Roulette}, in its current form, is only a
prototype, not something that can be deployed at scale. And even if the
software was improved to the extent that it was easy to use as
Minecraft, and even as successful, there is no guarantee that it would
be used in schools in more than an ad-hoc basis.

\subsection{What is needed for microworlds to succeed?}

I believe that microworlds cannot succeed until the two problems
mentioned above have been solved.

\begin{enumerate}[1.]
\item
  Teachers need to have the technical skills to construct and maintain
  microworlds.
\item
  There needs to be the political will to have a constructionist
  national curriculum instead of today's grade-focused curriculum.
\end{enumerate}

\subsection{When could this happen?}

Back in Section 2 we noted that ``by the age of 21, the average young
person will have spent as much time playing video games as they will
have spent in a classroom''. This means in the near future, we will have
young people entering the teaching profession who have extensive
knowledge of playing video games, or exploring microworlds. Hopefully,
this means the next generation of teachers will have the skills to
maintain virtual microworlds such as Minecraft.

The same applies for the world of politics. In the near future, we will
have young people entering parliament who have extensive knowledge of
exploring microworlds in the form of video games. Hopefully, this means
they will be better disposed toward constructionist learning, having
experienced it first-hand in video games.

However, many people today believe video games to be a waste of
time\{debate\}. Some gamers themselves believe that gaming is a waste of
time\{reddit\}. This suggests that having gamers in parliament may not
necessarily lead to more microworlds in the school curriculum.

Neither does having more gamers in teaching and politics mean that
microworlds are any easier to \emph{construct}. Constructing virtual
microworlds requires expertise in software engineering, as a virtual
microworld is a complex software artifact. Today, only developers with
industry experience have that expertise.

This is why I believe that it is only when the government employs
developers to construct microworlds, and allows teachers to use them,
will we see microworlds deployed in education at scale.

\subsection{Reality check}

Today, we are \textbf{optimising for administration}. We want work to be quantifiable, because numbers and grades are easiest to file. The more creative freedom a student has, the harder it is to reduce their work to a standardised grade. That's why microworlds are structurally incompatible with the goals of educational authorities today.

According to educational authorities today, the best work is formal, objective, rigorous and in a standardised format. 

That means the best test is a multiple-choice test. The best Master's thesis is one with a clear hypothesis and quantifiable results. A longitudinal study of answers to a multiple-choice survey, for example, or an analysis of the performance of different algorithms on the same dataset.

This is not that kind of Master's thesis. The microworlds created for this paper are original and innovative. The discussion of them is opinionated, and driven more by personal experience than academic research. 

My work is a result of knowledge I have personally constructed. I learned about microworlds by creating them and evaluating them, in a process that \textit{is itself a microworld}. The three parts of this thesis are themselves the three phases of the microworld loop.

This paper is qualitative rather than quantitative. It argues that learning is a process of iterative, informal experimentation, and is itself the product of iterative, informal experimentation. 

That makes this work hard to administrate. Which probably means that it is unacceptable as a Master's thesis, in it's current form.

That doesn't change the fact that I learned a great deal by creating microworlds, trying them on students, and then writing about my experiences. If you've made it this far, and this paper has made you think, then I'm satisfied, regardless of whether this thesis is accepted. 

\begin{center}\rule{3in}{0.4pt}\end{center}

\{ing:email\} Personal correspondence with Eileen Ingulfson, 13 March
2014 \{debate\}
http://www.debate.org/opinions/are-video-games-a-waste-of-time
\{reddit\}
\href{\%5D}{http://www.reddit.com/r/changemyview/comments/1ye5w2/i\_think\_that\_playing\_video\_games\_is\_a\_waste\_of/}
