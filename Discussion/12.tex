\section{Conclusions}

Microworlds seem to enable learning in an engaging and effective way, by
allowing students to construct knowledge through exploring rich environments.
Unfortunately, there does not seem to be infrastructure in place in
educational institutions today to fully embrace virtual microworlds, as teachers do
not have the technical skills required to create or maintain them.

Teachers at Kungsholmens Gymnasium thought \emph{Narrative Roulette} was
a great idea, but could not find lesson-time for me to hold workshops. I only
managed to hold my three workshops when I worked together with a teacher
to have them during substitute lessons, and only then when those
substitute lessons came up too fast for the substitute teacher to be
able to produce a lesson plan for them. It was only in this very
particular scenario, when lessons appeared in which the actual
curriculum could not be followed, that \emph{Narrative Roulette} was
given a chance.

The students loved \emph{Narrative Roulette}\cite{ingulfson}. As I have argued, they also
learned a lot by participating.

But only I could maintain the software during the workshops, and even if
I improved the interface so that others could run it, the workshops did
not provide clear, direct value for the current curriculum. This means
that \emph{Narrative Roulette}, in its current form, is only a
prototype, not something that can be deployed at scale. 
