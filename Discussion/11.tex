\section{What problems are there with microworlds?}

Let's imagine that microworlds - virtual, physical and hybrid - are
adopted in the mainstream curricula of schools and universities
worldwide. What consequences would this have for the structure of
education?

\subsection{Fully embracing microworlds}

As discussed in the previous section, lectures could remain relatively
unchanged, with the difference being that they are no longer mandatory,
but can be voluntarily attended by students undertaking self-directed
research. These lectures could still be labelled by subject in the same
way as today, for example: the biology of psychopathy is Natural
Science; the ethics of violence is a Social Science; and the literature
of crime is Art. The reasoning behind this is that the label
simply dictates the \emph{angle} the subject will be approached from. As
lectures are voluntary, students can mix and match these angles as they
wish.

Instead of producing work within a subject area such as Biology,
Philosophy or English Literature, students would then apply their
knowledge by exploring microworlds that span multiple subject areas.

Their aim would be to create public entities that can be peer-evaluated, by the
consumption of the work by a student's peers, and the sharing and
discussion it generates. A possible work could be creating an
interactive virtual environment, like a game, where players explore the
concept of psychopathy, from a biological, ethical and literary
perspective.

This could be something like a point-and-click adventure where you play
a psychopath, and perceive the world as a student believes a psychopath
would. The student who built the game would have to study biological,
psychological and literary work on psychopathy to create a realistic
world. The student could then explore ethical questions by offering
players choices about how to behave. The resulting narrative in the game
world would demonstrate students' literary abilities.

For variety, students could work on multiple projects in multiple
microworlds, all with different themes, at the same time during a term.
Every so often, a project would come to an end, and students' work would
be presented in an exhibition and peer-evaluated.

After one set of projects have been evaluated, a new set of projects, in
their own specifically-designed microworlds, would begin.

\subsection{From lesson plans to microworld construction}

If lessons are replaced with microworlds, those microworlds will have to
be constructed. Physical microworlds, like jamming with instruments or
creative writing on paper can be constructed by teachers today.

Virtual or hybrid microworlds are harder to construct without technical
skills. \emph{Talking to Machines} and \emph{Narrative Roulette} had to
be programmed specifically - they had too many custom requirements to be
put together from a standard system.

I believe I gained the majority of the skills required to program those
microworlds from working in the software industry, not from my Computer
Science education. This means that it takes software industry experience
to construct microworlds - something which the vast majority of teachers
do not currently have.

Even if the construction of microworlds were outsourced to the software
industry, teachers would need considerable experience with those
microworlds to help and advise students on how to best explore them. As
most microworlds are complex software artifacts, this is no
easy task.

\subsection{Transfer of control}

When students are allowed the freedom to have their own ideas and to choose
their modes of expression in a microworld, a lot of control is transferred from teacher (or school) to the student. If a teacher's performance is rated by the
grades of her students, that teacher may be understandably reluctant to
reduce her amount of control, as losing control over something means
increasing risk.

The same applies for politicians. If the performance of a department of
education is valued by global ratings, such as the PISA rankings, that
department will be understandably reluctant to transfer control to
teachers (who would then transfer control to students, according to the
microworld model). So instead, governments push for more exams, more
grades\cite{dn:bjorklund} and more inspections of educational institutions\cite{pedmag}. I believe this to be an unfortunate yet understandable reaction, given their position.

But as Hans-Åke Scherp writes in Pedagogiska Magasinet, 
``more control [by the government over schools] isn't solving the
problem''\cite{pedmag}.

\subsection{Keeping microworlds at arm's-length}

It seems that fully embracing microworlds is impractical for at least
two reasons: firstly, teachers do not have the necessary skills to
construct microworlds or to help students explore them, and secondly the
microworld model requires the transfer of control from the powerful
(politicians) to the powerless (teachers and students).

Luckily, that doesn't mean that microworlds can't be a small part of the
existing curriculum, sandwiched between traditional lectures, exercise
sessions and exams. Last year (2013), Viktor Rydberg Gymnasium in
Stockholm had a mandatory class centred on the Minecraft virtual
microworld. Monica Ekman, a teacher at Viktor Rydberg, has described the
experiment as a ``great success''\cite{local:minecraft}. This shows that microworlds
can be integrated into existing curricula successfully.

\subsection{The Minecraft model}

It's interesting to consider Minecraft as a case study in getting a
microworld into the classroom. At the time of the 2013 experiment at
Viktor Rydberg, Minecraft was a microworld with over 40 million users
and 17.5 million copies sold. I believe it was this scale that allowed
Minecraft to be taken seriously as an educational tool by schools, to
the extent that those schools spent considerable lesson-time on it.

That said, as of 2014, no other school has decided to allocate as much
resources to Minecraft as Viktor Rydberg, so it looks unlikely that
Minecraft will become part of any national curriculum in the near
future.

This suggests that having 40 million users spending hundreds of millions of hours
in your microworld doesn't guarantee that your microworld will make it
into most schools' curricula. This may be a result of the problems faced by 
microworlds outlined above.
