\section{Was Narrative Roulette successful?}

\emph{Talking to Machines} was a microworld. We saw from its structure
how it could enable learning, but we decided it was impractical to test
in a classroom setting. \emph{Narrative Roulette} kept the same
structure, but changed domain from programming languages to natural
language, to make the microworld more accessible.

\emph{Narrative Roulette} was engaging: there were
140 submissions by 60 students in three obligatory sessions at Kungsholmens Gymnasium. This means that 60 teenagers iterated the microworld
loop \emph{at least} 140 times in total.

Why? While writing, students must iterate the loop at a micro-level in
their own heads (or on their own screens): they have \textbf{ideas}; \textbf{implement}
them; read them back (\textbf{evaluating} them); and then update their ideas. 

The lower bound for a student's number of iterations of the loop is one per submission. This occurs if they do \emph{everything right the first time}. In that case,
they have an idea, implement it, and evaluate it to be perfect;
impossible to improve. Then they send in their submission. Given that this seems unlikely, it is safe to assume that most students iterate the loop multiple times
per submission.

\subsection{Did iterating the loop lead to learning?}

So is there any \textbf{evidence} that iterating the loop led to learning for the students taking part? 

Let's try and evaluate \emph{Narrative Roulette} submissions according to the framework set out in the previous chapter. 

The first question is: were texts read by students - did the texts fulfil their purpose? According to the behaviour tracking software I had on the site, \href{http://mixpanel.com}{Mixpanel}, the following submissions had significant reading activity:

\begin{center}
  \begin{tabular}{ | l | r | }
    \hline
    \textbf{Submission} & \textbf{Number of pageviews}\\ \hline
    \href{http://kg.narrativeroulette.com/submission/156}{156} & 229 \\
    \href{http://kg.narrativeroulette.com/submission/15}{15} & 114 \\
    \href{http://kg.narrativeroulette.com/submission/16}{16} & 105 \\
    \href{http://kg.narrativeroulette.com/submission/25}{25} & 52 \\
    \href{http://kg.narrativeroulette.com/submission/26}{26} & 45 \\
    \href{http://kg.narrativeroulette.com/submission/27}{27}  & 42 \\
    \hline
  \end{tabular}
\end{center}

This proves that students read each others' work, even though they didn't have to. They knew they weren't being graded for this exercise, and as I and the class teacher were the only adults in a room with 30 students, we couldn't keep an eye on all of them at the same time. They could have been doing other things on their laptops when we weren't looking at their screens.

But they didn't. The data corroborates what I saw - students were engaged in both reading and writing their texts. According to the framework set out in the previous section, this means that the texts fulfilled their purpose - they were read by others. 

They were also discussed in groups, orally, after each round. Our discussions were non-trivial: we talked about moral dilemmas, the use of language, characterisation, gender roles, intertextuality and more. 

That the work created by students, in the form of 140 narratives, generated reads and discussion, proves that the work was valuable, and the process that created that work was an effective learning experience.   
