\section{Was Narrative Roulette successful?}

\emph{Talking to Machines} was a microworld. We saw from its structure
how it could enable learning, but we decided it was impractical to test
in a classroom setting. \emph{Narrative Roulette} kept the same
structure, but changed domain from programming languages to natural
language, to make the microworld more accessible.

\emph{Narrative Roulette} was engaging: there were
140 submissions by 60 students in three obligatory sessions at Kungsholmens Gymnasium. This means that 60 teenagers iterated the microworld
loop \emph{at least} 140 times in total.

Why? While writing, students must iterate the loop at a micro-level in
their own heads (or on their own screens): they have \textbf{ideas}; \textbf{implement}
them; read them back (\textbf{evaluating} them); and then update their ideas. 

The lower bound for a student's number of iterations of the loop is one per submission. This occurs if they do \emph{everything right the first time}. In that case,
they have an idea, implement it, and evaluate it to be perfect;
impossible to improve. Then they send in their submission. Given that this seems unlikely, it is safe to assume that most students iterate the loop multiple times
per submission.

\subsection{Did iterating the loop lead to learning?}

We chose to evaluate Narrative Roulette by evaluating the quality of the discussion of student's texts. These discussions took place orally, after each round. 

Our discussions turned out to be non-trivial: we talked about moral dilemmas, the use of language, characterisation, gender roles, intertextuality and more \cite{ingulfson}. 

That the work created by students, in the form of 140 narratives, generated reads and discussion, proves that the work was valuable, and the process that created that work was an effective learning experience.   

\subsection{TODO: COUNTER ARGUMENTS GO HERE}
Yeah!
