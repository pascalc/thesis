\section{Was Narrative Roulette successful?}

\emph{Talking to Machines} was a microworld. We saw from its structure
how it could enable learning, but we decided it was impractical to test
in a classroom setting. \emph{Narrative Roulette} kept the same
structure, but changed domain from programming languages to natural
language, to make the microworld more accessible.

\emph{Narrative Roulette} was engaging: there were
140 submissions by 60 students in three obligatory sessions at Kungsholmens Gymnasium. This means that 60 teenagers iterated the microworld
loop \emph{at least} 140 times in total.

Why? While writing, students must iterate the loop at a micro-level in
their own heads (or on their own screens): they have \textbf{ideas}; \textbf{implement}
them; read them back (\textbf{evaluating} them); and then update their ideas. 

The lower bound for a student's number of iterations of the loop is one per submission. This occurs if they do \emph{everything right the first time}. In that case,
they have an idea, implement it, and evaluate it to be perfect;
impossible to improve. Then they send in their submission. Given that this seems unlikely, it is safe to assume that most students iterate the loop multiple times
per submission.

\subsection{Did iterating the loop lead to learning?}

We chose to evaluate Narrative Roulette by evaluating the quality of the discussion of student's texts. These discussions took place orally, after each round. 

Our discussions turned out to be non-trivial: we talked about moral dilemmas, the use of language, characterisation, gender roles, intertextuality and more\cite{ingulfson}. 

That the work created by students, in the form of 140 narratives, generated reads and intelligent discussion (as evaluated by myself and Eileen Ingulfson\cite{ingulfson}, a teacher with decades of experience), proves that the work was valuable, and the process that created that work was an effective learning experience.

\subsection{Arguments against constructionism}

There have been many counter arguments made against Piaget's constructivism, and a few (perhaps due to its recency) against Papert's constructionism.

Here are three, taken from Thirteen.org\cite{thirteen}:

\begin{enumerate}

\item ``It's elitist. Critics say that constructivism and other "progressive" educational theories have been most successful with children from privileged backgrounds who are fortunate in having outstanding teachers, committed parents, and rich home environments. They argue that disadvantaged children, lacking such resources, benefit more from more explicit instruction.''

\item ``Social constructivism leads to "group think." Critics say the collaborative aspects of constructivist classrooms tend to produce a "tyranny of the majority," in which a few students' voices or interpretations dominate the group's conclusions, and dissenting students are forced to conform to the emerging consensus.''

\item ``There is little hard evidence that constructivist methods work. Critics say that constructivists, by rejecting evaluation through testing and other external criteria, have made themselves unaccountable for their students' progress. Critics also say that studies of various kinds of instruction -- in particular Project Follow Through, a long-term government initiative -- have found that students in constructivist classrooms lag behind those in more traditional classrooms in basic skills.''

\end{enumerate} 

These are my thoughts on each of the above arguments, in the same order:

\begin{enumerate}

\item I cannot refute the argument about elitism. I conducted my workshops at Kungsholmens Gymnasium, one of the hardest schools to get into in the country. This suggests that many of the students that took my workshops came from privileged backgrounds. I was not able to try out \textit{Narrative Roulette} on a more diverse student population due to time constraints.

\item I tried to get around the ``tyranny of the majority'' by making students' texts anonymous. This ensured that students would discuss each others' texts in an unbiased way, but could not prevent oral group discussions potentially suffering from a ``tyranny of the majority''.

\item I have tried to account for the value of constructivist work in this very thesis. It is up to the reader (and more pertinently, my examiner) to say how well I have succeeded.

\end{enumerate} 
