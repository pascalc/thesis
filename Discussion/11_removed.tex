\subsection{Transfer of control}

When students are allowed the freedom to have their own ideas and to choose
their modes of expression in a microworld, a lot of control is transferred from teacher (or school) to the student. If a teacher's performance is rated by the
grades of her students, that teacher may be understandably reluctant to
reduce her amount of control, as losing control over something means
increasing risk.

The same applies for politicians. If the performance of a department of
education is valued by global ratings, such as the PISA rankings, that
department will be understandably reluctant to transfer control to
teachers (who would then transfer control to students, according to the
microworld model). So instead, governments push for more exams, more
grades\cite{dn:bjorklund} and more inspections of educational institutions\cite{pedmag}. I believe this to be an unfortunate yet understandable reaction, given their position.

But as Hans-Åke Scherp writes in Pedagogiska Magasinet, 
``more control [by the government over schools] isn't solving the
problem''\cite{pedmag}.

\subsection{Keeping microworlds at arm's-length}

It seems that fully embracing microworlds is impractical for at least
two reasons: firstly, teachers do not have the necessary skills to
construct microworlds or to help students explore them, and secondly the
microworld model requires the transfer of control from the powerful
(politicians) to the powerless (teachers and students).

Luckily, that doesn't mean that microworlds can't be a small part of the
existing curriculum, sandwiched between traditional lectures, exercise
sessions and exams. Last year (2013), Viktor Rydberg Gymnasium in
Stockholm had a mandatory class centred on the Minecraft virtual
microworld. Monica Ekman, a teacher at Viktor Rydberg, has described the
experiment as a ``great success''\cite{local:minecraft}. This shows that microworlds
can be integrated into existing curricula successfully.

\subsection{The Minecraft model}

It's interesting to consider Minecraft as a case study in getting a
microworld into the classroom. At the time of the 2013 experiment at
Viktor Rydberg, Minecraft was a microworld with over 40 million users
and 17.5 million copies sold. I believe it was this scale that allowed
Minecraft to be taken seriously as an educational tool by schools, to
the extent that those schools spent considerable lesson-time on it.

That said, as of 2014, no other school has decided to allocate as much
resources to Minecraft as Viktor Rydberg, so it looks unlikely that
Minecraft will become part of any national curriculum in the near
future.

This suggests that having 40 million users spending hundreds of millions of hours
in your microworld doesn't guarantee that your microworld will make it
into most schools' curricula. This may be a result of the problems faced by 
microworlds outlined above.
